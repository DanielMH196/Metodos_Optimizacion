
\documentclass{beamer}
\usepackage[utf8]{inputenc}
\usetheme{Madrid}
\usecolortheme{default}
\begin{document}

% Diapositiva 1
\begin{frame}
\frametitle{1. Introducción}
Una asociación tiene 100 hectáreas de terreno agrícola. La quinua rinde 2 toneladas por hectárea y la cañihua rinde 3 toneladas por hectárea. 
Por razones geográficas, la restricción del uso de terreno está dada por:
\[
x^2 + y^2 = 100
\]
donde $x$ es la cantidad de hectáreas dedicadas a quinua e $y$ a cañihua.\\

Se desea maximizar la producción total $f(x, y) = 2x + 3y$ bajo esta restricción.
\end{frame}

% Diapositiva 2
\begin{frame}
\frametitle{2. Planteamiento del Problema}
\begin{itemize}
    \item Función objetivo: $f(x, y) = 2x + 3y$
    \item Restricción: $x^2 + y^2 = 100$
    \item $x$: hectáreas de quinua, $y$: hectáreas de cañihua
    \item ¿Cómo maximizar la producción respetando el terreno?
\end{itemize}
\end{frame}

% Diapositiva 3
\begin{frame}
\frametitle{3. Método de Lagrange - Paso 1 y 2}
\textbf{Paso 1:} Lagrange:
\[
L(x, y, \lambda) = 2x + 3y - \lambda(x^2 + y^2 - 100)
\]
\vspace{0.4cm}
\textbf{Paso 2:} Derivadas parciales:
\[
\frac{\partial L}{\partial x} = 2 - 2\lambda x = 0 \quad 
\frac{\partial L}{\partial y} = 3 - 2\lambda y = 0 \quad 
\frac{\partial L}{\partial \lambda} = x^2 + y^2 - 100 = 0
\]
\end{frame}

% Diapositiva 4
\begin{frame}
\frametitle{4. Paso 3: Sistema de ecuaciones}
\begin{itemize}
    \item De $2 = 2\lambda x$ y $3 = 2\lambda y$, se obtiene:
    \[
    \frac{1}{x} = \frac{3}{2y} \Rightarrow y = \frac{3}{2}x
    \]
    \item Sustituimos en la restricción:
    \[
    x^2 + \left(\frac{3}{2}x\right)^2 = 100
    \]
    \[
    \frac{13}{4}x^2 = 100 \Rightarrow x^2 = \frac{400}{13}
    \Rightarrow x \approx 5.54, \quad y \approx 8.31
    \]
\end{itemize}
\end{frame}

% Diapositiva 5
\begin{frame}
\frametitle{5. Producción total}
\[
f(x, y) = 2x + 3y
\]
\[
f(5.54, 8.31) = 2(5.54) + 3(8.31) \approx 11.08 + 24.93 = \boxed{36.01\ \text{toneladas}}
\]
\end{frame}

% Diapositiva 6
\begin{frame}
\frametitle{6. Conclusión}
\begin{itemize}
    \item Se deben sembrar:
    \begin{itemize}
        \item 5.54 hectáreas de quinua
        \item 8.31 hectáreas de cañihua
    \end{itemize}
    \item Producción total: $\approx$ 36.01 toneladas
    \item Uso eficiente de las 100 hectáreas disponibles
\end{itemize}
\end{frame}

\end{document}
